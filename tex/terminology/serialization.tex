\sectionLabel{Data Serialization}

In computer science, in the context of data storage, serialization is the
process of translating data structures or object state into a format that can
be stored (for example, in a file or memory buffer, or transmitted across a
network connection) and reconstructed later in the same or another computer
environment. When the resulting series of bits is reread according to the
serialization format, it can be used to create a semantically identical clone
of the original object. For many complex objects, such as those that make
extensive use of references, this process is not straightforward. Serialization
of object-oriented objects does not include any of their associated methods
with which they were previously inextricably linked. This process of
serializing an object is also called \textit{marshalling} an object. The
opposite operation, extracting a data structure from a series of bytes, is
deserialization (which is also called \textit{unmarshalling}).

Data serialization allows to perform operations on test and training data
``in-memory'' instead of reading them from disk, thus increasing speed.
Serializing the data structure in an architecture independent format means
preventing the problems of byte ordering, memory layout, or simply different
ways of representing data structures in different programming languages.
Inherent to any serialization scheme is that, because the encoding of the data
is by definition serial, extracting one part of the serialized data structure
requires that the entire object be read from start to end, and reconstructed.
In applications where higher performance is an issue, it can make sense to
expend more effort to deal with a more complex, non-linear storage
organization. This algorithm uses an XML representation for the objects.
Several object-oriented programming languages directly support object
serialization (or object archival), either by syntactic sugar elements or
providing a standard interface for doing so.
\null\vfill

\addcontentsline{lot}{table}{Example \textit{XmlSerializer} serialization}
\begin{lstlisting}[language=xml]
<?xml version="1.0"?>
<ArrayOfSentence
 xmlns:xsi="http://www.w3.org/2001/XMLSchema-instance"
 xmlns:xsd="http://www.w3.org/2001/XMLSchema">
    <Sentence>
        <Words>
            <string>0%</string>
            <string>agarose</string>
            <string>gel</string>
        </Words>
        <PosTags>
            <string>JJ</string>
            <string>NN</string>
            <string>NN</string>
        </PosTags>
    </Sentence>
</ArrayOfSentence>
\end{lstlisting}

\addcontentsline{lot}{table}{Example \textit{DataContract} serialization}
\begin{lstlisting}[language=xml]
<ArrayOfComparator>
  <Comparator>
    <Cloud>
      <d3p1:KeyValueOfstringArrayOfanyTypety7Ep6D1>
        <d3p1:Key>council</d3p1:Key>
        <d3p1:Value>
          <d3p1:anyType>
            <d6p1:Demotion>0.5</d6p1:Demotion>
            <d6p1:Mistakes>457</d6p1:Mistakes>
            <d6p1:Promotion>1.5</d6p1:Promotion>
            <d6p1:Threshold>1</d6p1:Threshold>
            <d6p1:Weights>
              <d3p1:double>2.4892E-60</d3p1:double>
              <d3p1:double>6.5253E-55</d3p1:double>
              <d3p1:double>8.0779E-28</d3p1:double>
            </d6p1:Weights>
          </d3p1:anyType>
        </d3p1:Value>
      </d3p1:KeyValueOfstringArrayOfanyTypety7Ep6D1>
    </Cloud>
  </Comparator>
</ArrayOfComparator>
\end{lstlisting}
